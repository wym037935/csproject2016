\section{模块接口}
	\subsection{PC模块}
	\begin{table}[!hbp]
		\centering
		\begin{tabular}{|c|c|c|c|}
		\hline
		接口名&宽度&输入/输出&作用\\
		\hline
		rst &1& 输入& 复位信号\\
		\hline
		clk &1& 输入& 时钟信号\\
		\hline
		pc &32& 输出& 要读取的指令地址\\
		\hline
		ce &1& 输出& 指令存储器使能信号\\
		\hline
		branch{\_}flag{\_}i& 1& 输入& 是否转移\\
		\hline
		branch{\_}target{\_}address{\_}i& 32& 输入& 转移地址\\
		\hline
		new{\_}pc& 32& 输入& 要读取的指令地址\\
		\hline
		\end{tabular}
	\end{table}

	\subsection{Regfile模块}
	\begin{table}[!hbp]
		\centering
		\begin{tabular}{|c|c|c|c|}
		\hline
		接口名&宽度&输入/输出&作用\\
		\hline
		rst &1& 输入& 复位信号\\
		\hline
		clk &1& 输入& 时钟信号\\
		\hline
		waddr& 32& 输出& 要写入的寄存器地址\\
		\hline
		wdata& 1& 输出& 要写入的数据\\
		\hline
		we& 1& 输入& 写使能信号\\
		\hline
		raddr1& 5& 输入& 第一个读端口地址\\
		\hline
		re1& 1& 输入& 以一个读端口使能信号\\
		\hline
		rdata1& 32& 输出& 第一个读端口的值\\
		\hline
		raddr2& 5& 输入& 第二个读端口地址\\
		\hline
		re2& 1& 输入& 以二个读端口使能信号\\
		\hline
		rdata2& 32& 输出& 第二个读端口的值\\
		\hline
		\end{tabular}
	\end{table}
	
	\newpage
	\subsection{ID模块}
	\begin{table}[!hbp]
		\centering
		\begin{tabular}{|c|c|c|c|}
		\hline
		接口名&宽度&输入/输出&作用\\
		\hline
		rst &1& 输入& 复位信号\\
		\hline
		pc{\_}i& 32& 输入& 指令地址\\
		\hline
		inst{\_}i& 32& 输入& 译码阶段指令\\
		\hline
		reg1{\_}data{\_}i& 32& 输入& 第一个读端口输入\\
		\hline
		reg2{\_}data{\_}i& 32& 输入& 第二个读端口输入\\
		\hline
		reg1{\_}read{\_}o& 1& 输出& 第一个读端口使能信号\\
		\hline
		reg2{\_}read{\_}o& 1& 输出& 第二个读端口使能信号\\
		\hline
		reg1{\_}addr{\_}o& 5& 输出& 第一个读端口地址\\
		\hline
		reg2{\_}addr{\_}o& 5& 输出& 第二个读端口地址\\
		\hline
		aluop{\_}o& 8& 输出& 运算子类型\\
		\hline
		alusel{\_}o& 3& 输出& 运算类型\\
		\hline
		reg1{\_}o &32 &输出 &源操作数1\\
		\hline
		reg2{\_}o &32 &输出 &源操作数2\\
		\hline
		wd{\_}o& 5& 输出& 目的寄存器地址\\
		\hline
		wreg{\_}o& 1& 输出& 是否需要写入目的寄存器\\
		\hline
		ex{\_}wreg{\_}i& 1& 输入& 处于执行阶段指令是否写\\
		\hline
		ex{\_}wd{\_}i& 5& 输入& 处于执行阶段指令写地址\\
		\hline
		ex{\_}wdata{\_}i& 32& 输入& 处于执行阶段指令写数据\\
		\hline
		mem{\_}wreg{\_}i& 1& 输入& 处于访存阶段指令是否写\\
		\hline
		mem{\_}wd{\_}i& 5& 输入& 处于访存阶段指令写地址\\
		\hline
		mem{\_}wdata{\_}i& 32& 输入& 处于访存阶段指令写数据\\
		\hline
		branch{\_}flag{\_}o &1& 输出& 是否转移\\
		\hline
		branch{\_}target{\_}address{\_}o& 32& 输出& 转移目标地址\\
		\hline
		is{\_}in{\_}delayslot{\_}o &1& 输出& 当前指令是否位于延迟槽\\
		\hline
		link{\_}addr{\_}o &32 &输出& 返回地址\\
		\hline
		next{\_}inst{\_}in{\_}delayslot{\_}o &1 &输出& 下一跳指令是否位于延迟槽\\
		\hline
		is{\_}in{\_}delayslot{\_}i &1 &输入 &当前指令是否位于延迟槽\\
		\hline
		\end{tabular}
	\end{table}

\newpage
	\subsection{EX模块}
	\begin{table}[!hbp]
		\centering
		\begin{tabular}{|c|c|c|c|}
		\hline
		接口名&宽度&输入/输出&作用\\
		\hline
		rst &1& 输入& 复位信号\\
		\hline
		aluop{\_}i& 8& 输入& 运算子类型\\
		\hline
		alusel{\_}i& 3& 输入& 运算类型\\
		\hline
		reg1{\_}i &32 &输入 &源操作数1\\
		\hline
		reg2{\_}i &32 &输入 &源操作数2\\
		\hline
		wd{\_}i& 5& 输入& 目的寄存器地址\\
		\hline
		wreg{\_}i& 1& 输入& 是否需要写入目的寄存器\\
		\hline
		wd{\_}o& 5& 输出& 目的寄存器地址\\
		\hline
		wreg{\_}o& 1& 输出& 是否需要写入目的寄存器\\
		\hline
		wdata{\_}o& 32& 输出& 写入目的寄存器的值\\
		\hline
		is{\_}indelayslot{\_}i& 1& 输出& 是否位于延迟槽\\
		\hline
		link{\_}address{\_}i &32& 输出& 返回地址\\
		\hline
		mem{\_}addr{\_}o& 32& 输出& 加载/存储地址\\
		\hline
		reg2{\_}o& 32& 输出& 要存的数据\\
		\hline
		\end{tabular}
	\end{table}
	
	\subsection{MEM模块}
	\begin{table}[!hbp]
		\centering
		\begin{tabular}{|c|c|c|c|}
		\hline
		接口名&宽度&输入/输出&作用\\
		\hline
		rst &1& 输入& 复位信号\\
		\hline
		wd{\_}i& 5& 输入& 目的寄存器地址\\
		\hline
		wreg{\_}i& 1& 输入& 是否需要写入目的寄存器\\
		\hline
		wdata{\_}i& 32& 输入& 目的寄存器的值\\
		\hline
		wd{\_}o& 5& 输出& 目的寄存器地址\\
		\hline
		wreg{\_}o& 1& 输出& 是否需要写入目的寄存器\\
		\hline
		wdata{\_}o& 32& 输出& 写入目的寄存器的值\\
		\hline
		reg2{\_}i &32 &输出 &要存储的数据\\
		\hline
		mem{\_}data{\_}i& 32& 输入& 读取的数据\\
		\hline
		mem{\_}addr{\_}i& 32& 输入& 加载/存储地址\\
		\hline
		is{\_}write& 1& 输出& 是否写ram\\
		\hline
		\end{tabular}
	\end{table}