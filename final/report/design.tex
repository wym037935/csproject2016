\section{流水线设计}
	\subsection{取指阶段}
	\begin{itemize}
		\item PC 模块:给出指令地址, 其中实现指令寄存器 PC,该寄存器的值就是指令地址。
		\item IF/ID模块:实现取指不译码阶段之间的寄存器,将取指阶段的结果在下一个时钟传递到译码阶段。
	\end{itemize}
	
	
	\subsection{译码阶段}
	\begin{itemize}
		\item ID 模块:对指令进行译码,译码结果包括运算类型、运算所需的源操作数、要写入的目的寄存器等。
		\item Regfile 模块:实现了 32 个 32 位通用寄存器,可以同时进行两个寄存器的读操作和一个寄存器的写操作。
		\item ID/EX 模块:实现译码不执行阶段之间的寄存器,将译码阶段的结果在下一个时钟周期传递到执行阶段。
	\end{itemize}
	
	
	\subsection{执行阶段}
	\begin{itemize}
		\item EX 模块:依据译码阶段的结果,进行指定的运算,给出运算结果。
		\item EX/MEM 模块:实现执行不访存阶段之间的寄存器,将执行阶段的结果在下一个时钟周期传递到访存阶段。
	\end{itemize}
	
	
	\subsection{访存阶段}
	\begin{itemize}
		\item MEM 模块:如果是加载、存储指令,那么会对数据存储器进行访问。
		\item MEM/WB 模块:实现访存不回写阶段之间的寄存器,将访存阶段的结果在下一个时钟周期传递到回写阶段。
	\end{itemize}


	\subsection{回写阶段}
	\begin{itemize}
		\item HILO 模块:实现寄存器 HI、LO,在乘法指令的处理过程中会使用到这两个寄存器。
	\end{itemize}
	
	
\section{冲突问题}
	因为时间等因素,没有考虑冲突的问题,只是在fibonacci数列计算的程序中可能出现冲突的两条指令之间加上了4条NOP语句。