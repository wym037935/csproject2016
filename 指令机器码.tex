\documentclass[a4paper,UTF8,fntef]{ctexart}
\usepackage{graphicx}
\usepackage{amsmath,amssymb}
\usepackage{amscd}
\usepackage{latexsym}
\usepackage{arydshln,multirow}
\makeatletter %使\section中的内容左对齐
\renewcommand{\section}{\@startsection{section}{1}{0mm}
  {1.5\baselineskip}{0.5\baselineskip}{\bf\leftline}}
\newcommand{\tabincell}[2]{\begin{tabular}{@{}#1@{}}#2\end{tabular}}
\makeatother
\begin{document}
\title{\Large 指令机器码}
\author{计33~ ~伍一鸣~ ~2012011347\\计33~ ~杜华峰~ ~2013011354\\计33~ ~郭栋~ ~2013011334}
\maketitle
\tableofcontents
\newpage
\section{指令机器码}
rd,rs,rt均为寄存器
\subsection{逻辑操作}
	\begin{table}[!hbp]
		\centering
		\begin{tabular}{|c|c|c|c|c|c|c|}
		\hline
		\multirow{2}{*}{指令编码} & 31-26&25-21 & 20-16&15-11 &10-6 &5-0\\
		\cline{2-7} & 000000 & rs & rt & rd & 00000 & 100100 \\
		\hline
		指令格式&\multicolumn{6}{|l|}{AND rd rs rt}\\
		\hline		
		指令功能&\multicolumn{6}{|l|}{R[d] $\leftarrow$ R[s] \& R[t]}\\
		\hline		
		功能说明&\multicolumn{6}{|l|}{将rs 与rt 的值相与后的结果保存至rd 中}\\
		\hline
		\end{tabular}
	\end{table}
	\begin{table}[!hbp]
		\centering
		\begin{tabular}{|c|c|c|c|c|c|c|}
		\hline
		\multirow{2}{*}{指令编码} & 31-26&25-21 & 20-16&15-11 &10-6 &5-0\\
		\cline{2-7} & 000000 & rs & rt & rd & 00000 & 100101 \\
		\hline
		指令格式&\multicolumn{6}{|l|}{OR rd rs rt}\\
		\hline		
		指令功能&\multicolumn{6}{|l|}{R[d] $\leftarrow$ R[s] $|$ R[t]}\\
		\hline		
		功能说明&\multicolumn{6}{|l|}{将rs 与rt 的值相或后的结果保存至rd 中}\\
		\hline
		\end{tabular}
	\end{table}
	\begin{table}[!hbp]
		\centering
		\begin{tabular}{|c|c|c|c|c|c|c|}
		\hline
		\multirow{2}{*}{指令编码} & 31-26&25-21 & 20-16&15-11 &10-6 &5-0\\
		\cline{2-7} & 000000 & rs & rt & rd & 00000 & 100110 \\
		\hline
		指令格式&\multicolumn{6}{|l|}{XOR rd rs rt}\\
		\hline		
		指令功能&\multicolumn{6}{|l|}{R[d] $\leftarrow$ R[s] $\land$ R[t]}\\
		\hline		
		功能说明&\multicolumn{6}{|l|}{将rs 与rt 的值相异或后的结果保存至rd 中}\\
		\hline
		\end{tabular}
	\end{table}
	\begin{table}[!hbp]
		\centering
		\begin{tabular}{|c|c|c|c|c|c|c|}
		\hline
		\multirow{2}{*}{指令编码} & 31-26&25-21 & 20-16&15-11 &10-6 &5-0\\
		\cline{2-7} & 000000 & rs & rt & rd & 00000 & 100111 \\
		\hline
		指令格式&\multicolumn{6}{|l|}{NOR rd rs rt}\\
		\hline		
		指令功能&\multicolumn{6}{|l|}{R[d] $\leftarrow \sim$(R[s] $|$ R[t])}\\
		\hline		
		功能说明&\multicolumn{6}{|l|}{将rs 与rt 的值或非后的结果保存至rd 中}\\
		\hline
		\end{tabular}
	\end{table}
\newpage

	\begin{table}[!hbp]
		\centering
		\begin{tabular}{|c|c|c|c|c|c|c|}
		\hline
		\multirow{2}{*}{指令编码} & 31-26&25-21 & 20-16&15-11 &10-6 &5-0\\
		\cline{2-7} & 001100 & rs & rt & \multicolumn{3}{|c|}{immediate} \\
		\hline
		指令格式&\multicolumn{6}{|l|}{ANDI rt rs immediate}\\
		\hline		
		指令功能&\multicolumn{6}{|l|}{R[t] $\leftarrow$  R[s] \& Zero-extend(immediate)}\\
		\hline		
		功能说明&\multicolumn{6}{|l|}{将 rs 的值与立即数零扩展后相与的结果保存至rt 中}\\
		\hline
		\end{tabular}
	\end{table}
	\begin{table}[!hbp]
		\centering
		\begin{tabular}{|c|c|c|c|c|c|c|}
		\hline
		\multirow{2}{*}{指令编码} & 31-26&25-21 & 20-16&15-11 &10-6 &5-0\\
		\cline{2-7} & 001110 & rs & rt & \multicolumn{3}{|c|}{immediate} \\
		\hline
		指令格式&\multicolumn{6}{|l|}{XORI rt rs immediate}\\
		\hline		
		指令功能&\multicolumn{6}{|l|}{R[t] $\leftarrow$  R[s] $\land$ Zero-extend(immediate)}\\
		\hline		
		功能说明&\multicolumn{6}{|l|}{将 rs 的值与立即数零扩展后相异或的结果保存至rt 中}\\
		\hline
		\end{tabular}
	\end{table}
	\begin{table}[!hbp]
		\centering
		\begin{tabular}{|c|c|c|c|c|c|c|}
		\hline
		\multirow{2}{*}{指令编码} & 31-26&25-21 & 20-16&15-11 &10-6 &5-0\\
		\cline{2-7} & 001111 & 00000 & rt & \multicolumn{3}{|c|}{immediate} \\
		\hline
		指令格式&\multicolumn{6}{|l|}{LUI rt immediate}\\
		\hline		
		指令功能&\multicolumn{6}{|l|}{R[t] $\leftarrow$   immediate * 65536}\\
		\hline		
		功能说明&\multicolumn{6}{|l|}{将 16 位立即数放至 rt 的高 16 位中}\\
		\hline
		\end{tabular}
	\end{table}
	\begin{table}[!hbp]
		\centering
		\begin{tabular}{|c|c|c|c|c|c|c|}
		\hline
		\multirow{2}{*}{指令编码} & 31-26&25-21 & 20-16&15-11 &10-6 &5-0\\
		\cline{2-7} & 001101 & rs & rt & \multicolumn{3}{|c|}{immediate} \\
		\hline
		指令格式&\multicolumn{6}{|l|}{ORI rt rs immediate}\\
		\hline		
		指令功能&\multicolumn{6}{|l|}{R[t] $\leftarrow$  R[s] $|$ Zero-extend(immediate)}\\
		\hline		
		功能说明&\multicolumn{6}{|l|}{将 rs 与立即数 immediate 零扩展后相或的结果保存至rd 中}\\
		\hline
		\end{tabular}
	\end{table}
\newpage

\subsection{移位操作}
	\begin{table}[!hbp]
		\centering
		\begin{tabular}{|c|c|c|c|c|c|c|}
		\hline
		\multirow{2}{*}{指令编码} & 31-26&25-21 & 20-16&15-11 &10-6 &5-0\\
		\cline{2-7} & 000000 & 00000 & rt & rd & immediate& 000000 \\
		\hline
		指令格式&\multicolumn{6}{|l|}{SLL rd rt immediate}\\
		\hline		
		指令功能&\multicolumn{6}{|l|}{R[d] $\leftarrow$  R[t] $<<$ immediate}\\
		\hline		
		功能说明&\multicolumn{6}{|l|}{将 rt 中的值左移立即数 immediate 位后的结果保存至 rd 中}\\
		\hline
		\end{tabular}
	\end{table}
	\begin{table}[!hbp]
		\centering
		\begin{tabular}{|c|c|c|c|c|c|c|}
		\hline
		\multirow{2}{*}{指令编码} & 31-26&25-21 & 20-16&15-11 &10-6 &5-0\\
		\cline{2-7} & 000000 & 00000 & rt & rd & immediate& 000010 \\
		\hline
		指令格式&\multicolumn{6}{|l|}{SRL rd rt immediate}\\
		\hline		
		指令功能&\multicolumn{6}{|l|}{R[d] $\leftarrow$  R[t] $>>$ immediate(logical)}\\
		\hline		
		功能说明&\multicolumn{6}{|l|}{将 rt 中的值逻辑右移立即数 immediate 位后的结果保存至 rd 中}\\
		\hline
		\end{tabular}
	\end{table}
	\begin{table}[!hbp]
		\centering
		\begin{tabular}{|c|c|c|c|c|c|c|}
		\hline
		\multirow{2}{*}{指令编码} & 31-26&25-21 & 20-16&15-11 &10-6 &5-0\\
		\cline{2-7} & 000000 & 00000 & rt & rd & immediate& 000011 \\
		\hline
		指令格式&\multicolumn{6}{|l|}{SRA rd rt immediate}\\
		\hline		
		指令功能&\multicolumn{6}{|l|}{R[d] $\leftarrow$  R[t] $>>$ immediate(arithmetic)}\\
		\hline		
		功能说明&\multicolumn{6}{|l|}{将 rt 中的值算术右移立即数 immediate 位后的结果保存至 rd 中}\\
		\hline
		\end{tabular}
	\end{table}
\newpage

	\begin{table}[!hbp]
		\centering
		\begin{tabular}{|c|c|c|c|c|c|c|}
		\hline
		\multirow{2}{*}{指令编码} & 31-26&25-21 & 20-16&15-11 &10-6 &5-0\\
		\cline{2-7} & 000000 & rs & rt & rd & 00000& 000100 \\
		\hline
		指令格式&\multicolumn{6}{|l|}{SLLV rd rt rs}\\
		\hline		
		指令功能&\multicolumn{6}{|l|}{R[d] $\leftarrow$  R[t] $<<$ R[s]}\\
		\hline		
		功能说明&\multicolumn{6}{|l|}{将 rt 中的值左移 rs 位后的结果保存至 rd 中}\\
		\hline
		\end{tabular}
	\end{table}
	\begin{table}[!hbp]
		\centering
		\begin{tabular}{|c|c|c|c|c|c|c|}
		\hline
		\multirow{2}{*}{指令编码} & 31-26&25-21 & 20-16&15-11 &10-6 &5-0\\
		\cline{2-7} & 000000 & rs & rt & rd & 00000& 000110 \\
		\hline
		指令格式&\multicolumn{6}{|l|}{SRLV rd rt rs}\\
		\hline		
		指令功能&\multicolumn{6}{|l|}{R[d] $\leftarrow$  R[t] $>>$ R[s](logical)}\\
		\hline		
		功能说明&\multicolumn{6}{|l|}{将 rt 中的值逻辑右移rs 位后的结果保存至 rd 中}\\
		\hline
		\end{tabular}
	\end{table}
	\begin{table}[!hbp]
		\centering
		\begin{tabular}{|c|c|c|c|c|c|c|}
		\hline
		\multirow{2}{*}{指令编码} & 31-26&25-21 & 20-16&15-11 &10-6 &5-0\\
		\cline{2-7} & 000000 & rs & rt & rd & 00000& 000111 \\
		\hline
		指令格式&\multicolumn{6}{|l|}{SRAV rd rt rs}\\
		\hline		
		指令功能&\multicolumn{6}{|l|}{R[d] $\leftarrow$  R[t] $>>$ R[s](arithmetic)}\\
		\hline		
		功能说明&\multicolumn{6}{|l|}{将 rt 中的值算术右移rs位后的结果保存至 rd 中}\\
		\hline
		\end{tabular}
	\end{table}
\newpage

\subsection{移动操作}
	\begin{table}[!hbp]
		\centering
		\begin{tabular}{|c|c|c|c|c|c|c|}
		\hline
		\multirow{2}{*}{指令编码} & 31-26&25-21 & 20-16&15-11 &10-6 &5-0\\
		\cline{2-7} & 000000 & rs & rt & rd & 00000& 001011 \\
		\hline
		指令格式&\multicolumn{6}{|l|}{MOVN rd rt rs}\\
		\hline		
		指令功能&\multicolumn{6}{|l|}{if rt $\neq$ 0 then rd $\leftarrow$ rs}\\
		\hline		
		功能说明&\multicolumn{6}{|l|}{若rt不为0,则将rs的值赋给rd}\\
		\hline
		\end{tabular}
	\end{table}
	\begin{table}[!hbp]
		\centering
		\begin{tabular}{|c|c|c|c|c|c|c|}
		\hline
		\multirow{2}{*}{指令编码} & 31-26&25-21 & 20-16&15-11 &10-6 &5-0\\
		\cline{2-7} & 000000 & rs & rt & rd & 00000& 001010 \\
		\hline
		指令格式&\multicolumn{6}{|l|}{MOVZ rd rt rs}\\
		\hline		
		指令功能&\multicolumn{6}{|l|}{if rt = 0 then rd $\leftarrow$ rs}\\
		\hline		
		功能说明&\multicolumn{6}{|l|}{若rt为0,则将rs的值赋给rd}\\
		\hline
		\end{tabular}
	\end{table}
\newpage

\subsection{算术操作}
	\begin{table}[!hbp]
		\centering
		\begin{tabular}{|c|c|c|c|c|c|c|}
		\hline
		\multirow{2}{*}{指令编码} & 31-26&25-21 & 20-16&15-11 &10-6 &5-0\\
		\cline{2-7} & 000000 & rs & rt & rd & 00000 & 100001 \\
		\hline
		指令格式&\multicolumn{6}{|l|}{ADDU rd rs rt}\\
		\hline		
		指令功能&\multicolumn{6}{|l|}{R[d] $\leftarrow$ R[s] + R[t]}\\
		\hline		
		功能说明&\multicolumn{6}{|l|}{将rs 与rt 的值相加后的结果保存至rd 中}\\
		\hline
		\end{tabular}
	\end{table}
	\begin{table}[!hbp]
		\centering
		\begin{tabular}{|c|c|c|c|c|c|c|}
		\hline
		\multirow{2}{*}{指令编码} & 31-26&25-21 & 20-16&15-11 &10-6 &5-0\\
		\cline{2-7} & 000000 & rs & rt & rd & 00000 & 100011 \\
		\hline
		指令格式&\multicolumn{6}{|l|}{SUBU rd rs rt}\\
		\hline		
		指令功能&\multicolumn{6}{|l|}{R[d] $\leftarrow$ R[s] - R[t]}\\
		\hline		
		功能说明&\multicolumn{6}{|l|}{将rs 与rt 的值相减后的结果保存至rd 中}\\
		\hline
		\end{tabular}
	\end{table}
	\begin{table}[!hbp]
		\centering
		\begin{tabular}{|c|c|c|c|c|c|c|}
		\hline
		\multirow{2}{*}{指令编码} & 31-26&25-21 & 20-16&15-11 &10-6 &5-0\\
		\cline{2-7} & 000000 & rs & rt & rd & 00000 & 101010 \\
		\hline
		指令格式&\multicolumn{6}{|l|}{SLT rd rs rt}\\
		\hline		
		指令功能&\multicolumn{6}{|l|}{if(R[s] $<$ R[t]) then R[d] = 1,else R[d] = 0}\\
		\hline		
		功能说明&\multicolumn{6}{|l|}{比较 rs 与 rt 的值并根据结果将 rd 赋值}\\
		\hline
		\end{tabular}
	\end{table}
	\begin{table}[!hbp]
		\centering
		\begin{tabular}{|c|c|c|c|c|c|c|}
		\hline
		\multirow{2}{*}{指令编码} & 31-26&25-21 & 20-16&15-11 &10-6 &5-0\\
		\cline{2-7} & 000000 & rs & rt & rd & 00000 & 101011 \\
		\hline
		指令格式&\multicolumn{6}{|l|}{SLTU rd rs rt}\\
		\hline		
		指令功能&\multicolumn{6}{|l|}{if(R[s] $<$ R[t]) then R[d] = 1,else R[d] = 0}\\
		\hline		
		功能说明&\multicolumn{6}{|l|}{比较 rs 与 rt 的无符号值并根据结果将 rd 赋值}\\
		\hline
		\end{tabular}
	\end{table}
\newpage

	\begin{table}[!hbp]
		\centering
		\begin{tabular}{|c|c|c|c|c|c|c|}
		\hline
		\multirow{2}{*}{指令编码} & 31-26&25-21 & 20-16&15-11 &10-6 &5-0\\
		\cline{2-7} & 001001 & rs & rt & \multicolumn{3}{|c|}{immediate} \\
		\hline
		指令格式&\multicolumn{6}{|l|}{ADDIU rt rs immediate}\\
		\hline		
		指令功能&\multicolumn{6}{|l|}{R[t] $\leftarrow$ R[s] + (sign extended)immediate}\\
		\hline		
		功能说明&\multicolumn{6}{|l|}{对立即数immediate进行符号扩展后与rs的值求和,保存到rt中,不检查溢出}\\
		\hline
		\end{tabular}
	\end{table}
	\begin{table}[!hbp]
		\centering
		\begin{tabular}{|c|c|c|c|c|c|c|}
		\hline
		\multirow{2}{*}{指令编码} & 31-26&25-21 & 20-16&15-11 &10-6 &5-0\\
		\cline{2-7} & 001010 & rs & rt & \multicolumn{3}{|c|}{immediate} \\
		\hline
		指令格式&\multicolumn{6}{|l|}{SLTI rt rs immediate}\\
		\hline		
		指令功能&\multicolumn{6}{|l|}{if( R[s] $<$ (sign extended)immediate )then R[t]=1,else R[t]=0}\\
		\hline		
		功能说明&\multicolumn{6}{|l|}{对立即数immediate进行符号扩展后与rs的值无符号比较并根据结果将 rt 赋值}\\
		\hline
		\end{tabular}
	\end{table}
	\begin{table}[!hbp]
		\centering
		\begin{tabular}{|c|c|c|c|c|c|c|}
		\hline
		\multirow{2}{*}{指令编码} & 31-26&25-21 & 20-16&15-11 &10-6 &5-0\\
		\cline{2-7} & 001011 & rs & rt & \multicolumn{3}{|c|}{immediate} \\
		\hline
		指令格式&\multicolumn{6}{|l|}{SLTIU rt rs immediate}\\
		\hline		
		指令功能&\multicolumn{6}{|l|}{if( R[s] $<$ (sign extended)immediate )then R[t]=1,else R[t]=0}\\
		\hline		
		功能说明&\multicolumn{6}{|l|}{对立即数immediate进行符号扩展后与rs的值有符号比较并根据结果将 rt 赋值}\\
		\hline
		\end{tabular}
	\end{table}
\newpage

\subsection{转移指令}
	\begin{table}[!hbp]
		\centering
		\begin{tabular}{|c|c|c|c|c|c|c|}
		\hline
		\multirow{2}{*}{指令编码} & 31-26&25-21 & 20-16&15-11 &10-6 &5-0\\
		\cline{2-7} & 000000 & rs & 00000 & 00000& 00000& 001000 \\
		\hline
		指令格式&\multicolumn{6}{|l|}{JR rs}\\
		\hline		
		指令功能&\multicolumn{6}{|l|}{PC $\leftarrow$ R[s]}\\
		\hline		
		功能说明&\multicolumn{6}{|l|}{无条件跳转至 rs 中所存地址执行}\\
		\hline
		\end{tabular}
	\end{table}
	\begin{table}[!hbp]
		\centering
		\begin{tabular}{|c|c|c|c|c|c|c|}
		\hline
		\multirow{2}{*}{指令编码} & 31-26&25-21 & 20-16&15-11 &10-6 &5-0\\
		\cline{2-7} & 000000 & rs & 00000 & rd& 00000& 001001 \\
		\hline
		指令格式&\multicolumn{6}{|l|}{JALR rd rs 或者JALR rs}\\
		\hline		
		指令功能&\multicolumn{6}{|l|}{PC $\leftarrow$ R[s], R[d] $\leftarrow$ RPC}\\
		\hline		
		功能说明&\multicolumn{6}{|l|}{\tabincell{c}{无条件跳转至 rs 中所存地址执行,将延时槽后一条指令\\的地址保存到 rd中作为返回地址,rd默认为\$31}}\\
		\hline
		\end{tabular}
	\end{table}
	\begin{table}[!hbp]
		\centering
		\begin{tabular}{|c|c|c|c|c|c|c|}
		\hline
		\multirow{2}{*}{指令编码} & 31-26&25-21 & 20-16&15-11 &10-6 &5-0\\
		\cline{2-7} & 000010 & \multicolumn{5}{|c|}{instr index} \\
		\hline
		指令格式&\multicolumn{6}{|l|}{J target}\\
		\hline		
		指令功能&\multicolumn{6}{|l|}{PC $\leftarrow$ (PC+4)[31,28]$||$target*4}\\
		\hline		
		功能说明&\multicolumn{6}{|l|}{\tabincell{c}{跳转至新地址执行,新地址低28位为target乘以4的值,\\新地址高4位为PC+4的高4位}}\\
		\hline
		\end{tabular}
	\end{table}
	\begin{table}[!hbp]
		\centering
		\begin{tabular}{|c|c|c|c|c|c|c|}
		\hline
		\multirow{2}{*}{指令编码} & 31-26&25-21 & 20-16&15-11 &10-6 &5-0\\
		\cline{2-7} & 000011 & \multicolumn{5}{|c|}{instr index} \\
		\hline
		指令格式&\multicolumn{6}{|l|}{JAL target}\\
		\hline		
		指令功能&\multicolumn{6}{|l|}{PC $\leftarrow$ (PC+4)[31,28]$||$target*4,\$31 $\leftarrow$ RPC}\\
		\hline		
		功能说明&\multicolumn{6}{|l|}{\tabincell{c}{跳转至新地址执行,新地址低28位为target乘以4的值,\\新地址高4位为PC+4的高4位,返回地址保存到\$31中}}\\
		\hline
		\end{tabular}
	\end{table}
	
	
	以上四条指令都要在转移之前先执行延迟槽指令
\newpage

	\begin{table}[!hbp]
		\centering
		\begin{tabular}{|c|c|c|c|c|c|c|}
		\hline
		\multirow{2}{*}{指令编码} & 31-26&25-21 & 20-16&15-11 &10-6 &5-0\\
		\cline{2-7} & 000100 & rs & rt & \multicolumn{3}{|c|}{offset} \\
		\hline
		指令格式&\multicolumn{6}{|l|}{BEQ rs rt offset}\\
		\hline		
		指令功能&\multicolumn{6}{|l|}{if (rs = rt) then PC = PC+4+(signed extend(offset * 4))}\\
		\hline		
		功能说明&\multicolumn{6}{|l|}{若rs等于rt则执行跳转操作}\\
		\hline
		\end{tabular}
	\end{table}
%	\begin{table}[!hbp]
%		\centering
%		\begin{tabular}{|c|c|c|c|c|c|c|}
%		\hline
%		\multirow{2}{*}{指令编码} & 31-26&25-21 & 20-16&15-11 &10-6 &5-0\\
%		\cline{2-7} & 000100 & 00000 & 00000 & \multicolumn{3}{|c|}{offset} \\
%		\hline
%		指令格式&\multicolumn{6}{|l|}{B offset}\\
%		\hline		
%		指令功能&\multicolumn{6}{|l|}{PC = PC+4+(signed extend(offset * 4))}\\
%		\hline		
%		功能说明&\multicolumn{6}{|l|}{无条件执行跳转操作}\\
%		\hline
%		\end{tabular}
%	\end{table}
	\begin{table}[!hbp]
		\centering
		\begin{tabular}{|c|c|c|c|c|c|c|}
		\hline
		\multirow{2}{*}{指令编码} & 31-26&25-21 & 20-16&15-11 &10-6 &5-0\\
		\cline{2-7} & 000111 & rs & 00000 & \multicolumn{3}{|c|}{offset} \\
		\hline
		指令格式&\multicolumn{6}{|l|}{BGTZ rs offset}\\
		\hline		
		指令功能&\multicolumn{6}{|l|}{if (rs $>$ 0) then PC = PC+4+(signed extend(offset * 4))}\\
		\hline		
		功能说明&\multicolumn{6}{|l|}{若rs大于0则执行跳转操作}\\
		\hline
		\end{tabular}
	\end{table}
	\begin{table}[!hbp]
		\centering
		\begin{tabular}{|c|c|c|c|c|c|c|}
		\hline
		\multirow{2}{*}{指令编码} & 31-26&25-21 & 20-16&15-11 &10-6 &5-0\\
		\cline{2-7} & 000110 & rs & 00000 & \multicolumn{3}{|c|}{offset} \\
		\hline
		指令格式&\multicolumn{6}{|l|}{BLEZ rs offset}\\
		\hline		
		指令功能&\multicolumn{6}{|l|}{if (rs $\leq$ 0) then PC = PC+4+(signed extend(offset * 4))}\\
		\hline		
		功能说明&\multicolumn{6}{|l|}{若rs不大于0则执行跳转操作}\\
		\hline
		\end{tabular}
	\end{table}
%\newpage

	\begin{table}[!hbp]
		\centering
		\begin{tabular}{|c|c|c|c|c|c|c|}
		\hline
		\multirow{2}{*}{指令编码} & 31-26&25-21 & 20-16&15-11 &10-6 &5-0\\
		\cline{2-7} & 000101 & rs & rt & \multicolumn{3}{|c|}{offset} \\
		\hline
		指令格式&\multicolumn{6}{|l|}{BNE rs rt offset}\\
		\hline		
		指令功能&\multicolumn{6}{|l|}{if (rs $\neq$ rt) then PC = PC+4+(signed extend(offset * 4))}\\
		\hline		
		功能说明&\multicolumn{6}{|l|}{若rs不等于rt则执行跳转操作}\\
		\hline
		\end{tabular}
	\end{table}
	\begin{table}[!hbp]
		\centering
		\begin{tabular}{|c|c|c|c|c|c|c|}
		\hline
		\multirow{2}{*}{指令编码} & 31-26&25-21 & 20-16&15-11 &10-6 &5-0\\
		\cline{2-7} & 000001 & rs & 00000 & \multicolumn{3}{|c|}{offset} \\
		\hline
		指令格式&\multicolumn{6}{|l|}{BLTZ rs offset}\\
		\hline		
		指令功能&\multicolumn{6}{|l|}{if (rs $<$ 0) then PC = PC+4+(signed extend(offset * 4))}\\
		\hline		
		功能说明&\multicolumn{6}{|l|}{若rs小于0则执行跳转操作}\\
		\hline
		\end{tabular}
	\end{table}
	\begin{table}[!hbp]
		\centering
		\begin{tabular}{|c|c|c|c|c|c|c|}
		\hline
		\multirow{2}{*}{指令编码} & 31-26&25-21 & 20-16&15-11 &10-6 &5-0\\
		\cline{2-7} & 000001 & rs & 00001 & \multicolumn{3}{|c|}{offset} \\
		\hline
		指令格式&\multicolumn{6}{|l|}{BLEZ rs offset}\\
		\hline		
		指令功能&\multicolumn{6}{|l|}{if (rs $\geq$ 0) then PC = PC+4+(signed extend(offset * 4))}\\
		\hline		
		功能说明&\multicolumn{6}{|l|}{若rs不小于0则执行跳转操作}\\
		\hline
		\end{tabular}
	\end{table}
\newpage
\subsection{存储指令}
	\begin{table}[!hbp]
		\centering
		\begin{tabular}{|c|c|c|c|c|c|c|}
		\hline
		\multirow{2}{*}{指令编码} & 31-26&25-21 & 20-16&15-11 &10-6 &5-0\\
		\cline{2-7} & 100011 & base & rt & \multicolumn{3}{|c|}{offset} \\
		\hline
		指令格式&\multicolumn{6}{|l|}{LW rt offset(base)}\\
		\hline		
		指令功能&\multicolumn{6}{|l|}{R[t] $\leftarrow$ MEM[signed extended(offset)+GPR[base]]}\\
		\hline		
		功能说明&\multicolumn{6}{|l|}{从内存中指定的加载地址处,读取一个字,保存到rt中,要求地址对齐}\\
		\hline
		\end{tabular}
	\end{table}
	\begin{table}[!hbp]
		\centering
		\begin{tabular}{|c|c|c|c|c|c|c|}
		\hline
		\multirow{2}{*}{指令编码} & 31-26&25-21 & 20-16&15-11 &10-6 &5-0\\
		\cline{2-7} & 101011 & base & rt & \multicolumn{3}{|c|}{offset} \\
		\hline
		指令格式&\multicolumn{6}{|l|}{SW rt offset(base)}\\
		\hline		
		指令功能&\multicolumn{6}{|l|}{R[t] $\rightarrow$ MEM[signed extended(offset)+GPR[base]]}\\
		\hline		
		功能说明&\multicolumn{6}{|l|}{从rt处读取一个字,保存到内存中指定的加载地址中,要求地址对齐}\\
		\hline
		\end{tabular}
	\end{table}
\subsection{空指令}
	\begin{table}[!hbp]
		\centering
		\begin{tabular}{|c|c|c|c|c|c|c|}
		\hline
		\multirow{2}{*}{指令编码} & 31-26&25-21 & 20-16&15-11 &10-6 &5-0\\
		\cline{2-7} & 000000 & 00000 & 00000 & 00000& 00000& 000000 \\
		\hline
		指令格式&\multicolumn{6}{|l|}{NOP}\\
		\hline		
		指令功能&\multicolumn{6}{|l|}{无}\\
		\hline		
		功能说明&\multicolumn{6}{|l|}{空指令}\\
		\hline
		\end{tabular}
	\end{table}
\end{document}